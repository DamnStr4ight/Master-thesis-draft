\newpage
\section{Radio Frequency Authorities}

When designing RF solutions it is important to be aware of the regulations of such systems. There are a few large institutions that governs the use of the radio frequencies in the world. Nasjonal kommunikasjonsmyndighet (Nkom) is what is found in Norway.

\begin{labeling}{ETSI}
\item[FCC] Federal Communications Commission
\item[ERO] European Radicommunications Office
\item[CEPT] European Conference of Postal and Telecommunications Administrations
\item[ETSI] European Telecommunications Standards Institute
\item[Nkom] Nasjonal kommunikasjonsmyndighet
\end{labeling}

Radio frequencies are technically a limited resource, even though you could get frequencies well beyond $10^{19}$ such extremely high frequencies are currently not viable for sending information due to lack of equipment.\\

Nkom has for this reason chosen to regulate the frequencies between 8.3 kHz to 3000 GHz. \cite{Frekvensplan} In §4 of the free use regulation states that frequencies outside these two are free to use so long it does not interfere with other communication. Some well known bands within the regulated range are also free to use, like the two most used WiFi bands around 2.4 GHz and 5.8 GHz.\\

Apart from 20kHz, all frequencies below 30MHz are open for inductive applications. When designing an RFID system with a coil as an antenna, it falls within this category. This mean that a designer is free to tune the system to whichever frequencies one might like within this spectrum.\\

While they are free to use, the law dictates one property of a system, its field strength at 10m distance. for example, the maximum field strength at 10m for 125kHz is 66dB$\mu$A/m.

