\newpage
\section{Internet of Things}
The Internet of Things is a network of connected devices who traditionally were offline. These devices are cars, home appliances, embedded electronics such as light bulbs, temperature sensors, actuators and similar devices. The developer in the field aim to make the life of the general public easier, by making home automation easy, and to make the world more interconnected. It is estimated that by the year 2020, there will be some 30 billion devices connected to this network. \cite{nordrum2016popular}\\

A research network called Auto-ID Labs is currently working on an IoT RFID solution. They have members in seven of the worlds most renowned research universities. They aim to find new ways to benefit the global commerce market and provide previously realizable consumer benefits.\\

A big concern with IoT is that many implementations lack sufficient security. For example if one scans an MQTT (see section \ref{sec:MQTT}) network, a common communication protocol, it is not uncommon to find all sorts of light bulbs, temperature sensors and the like. This is a danger that many people do not realize, as a potential burglar might use the information from these sources to figure out behaviour patterns and know when a break-in is more likely to succeed.\\

Another large concern is the fact that if a sufficient amount of devices in an area is hacked, and can be turned on at the same time, it could potentially overload and fry the power grid. On a large enough scale, this has the potential to black out entire regions, if not countries.
