\chapter*{Assignment text}
\subsection*{Demo: IoT Chess Boards}

Thesis description: Radio-frequency identification (RFID) technology is used to identify and track tags attached to objects. The tags, which have no stored energy contain electronically-stored information, typically examples of use are ID cards and labels on products. The RFID technology can also be used in sensors for IoT (Internet of Things) applications, to sense the presence of other objects with RFID tags. 
\newline\newline
This thesis aims to develop a “show-case”, demonstrating the ATtiny1617 MCU in a RFID system together with the ATmega4809 and WINC1500 to set up an IoT node. Using the analog capabilities of the ATtiny1617, the candidate should develop a BOM optimized implementation of the RFID reader. This should be done by embedding these technologies in a chessboard and the chess pieces, to make remote chess and to play chess against computers possible.
\newline\newline
The tasks will be:
\begin{enumerate}
    \item Conduct a literary study and elaborate on the the RFID technology in general and with an especial focus on controlling the range and how to use the technology for position decision of objects.
    \item Design a RFID reader which is BOM optimized.
    \item Design a system to monitor the positions and movements of the chess pieces on a chessboard.
    \item As far as time permits, implement the suggested monitoring system.
\end{enumerate}
 
